\documentclass[a4paper]{article}

\usepackage{polski}
\usepackage[utf8]{inputenc}
\usepackage[polish]{babel}
\usepackage[pdftex]{graphicx}
\usepackage[pdftex]{hyperref}
\usepackage{mdwlist}

\title{Specyfikacja wymagań dla systemu telemedycznego realizowanego w ramach projektu XXX}
\author{Zespół Projektowy}
\date{03/03/2012}

\begin{document}

\maketitle
\tableofcontents

Opis funkcjonalny Systemu

\section{Wstęp}

Od strony funkcjonalnej System składa się z dwóch niezależnych podsystemów działających na tej samej platformie sprzętowej i programowej: Podsystem Elektronicznego Zlecania Opisów Badań Medycznych (zwany danej Systemem Zlecania) oraz Podsystem Elektronicznej Wymiany Danych Medycznych (zwany dalej Bazą Metadanych).

\subsection{Definicje}

\begin{description}
\item[System] centralny system telemedyczny, będący przedmiotem tej specyfikacji
\item[Zleceniodawca] użytkownik węzła lokalnego zlecający do systemu wykonanie opisu badania
\item[Konsultant] użytkownik węzła lokalnego realizujący zlecenie opisu badania, które pojawiło się na jego węźle
\item[Zlecenie] zlecenie opisu badania medycznego
\item[Węzeł centralny] serwer lub grupa serwerów znajdująca się w siedzibie zamawiającego; węzeł centralny posiada informacje o węzłach lokalnych i ma stałą łączność z nimi
\item[Węzeł lokalny] serwer znajdujący się w siedzibie użytkownika (w szpitalu); węzeł lokalny ma łączność z węzłem centralnym; innymi węzłami lokalnymi oraz z LSS
\item[LSS] lokalne systemy szpitalne -- wewnętrzne systemy funkcjonujące w danej jednostce medycznej, typu PACS, HIS, RIS; komunikacja Systemu z nimi odbywa się za pomocą protokołów DICOM oraz HL7
\end{description}

\section{Założenia wspólne dla obu podsystemów}

\subsection{Architektura fizyczna}
\label{sec:arch_fiz}

System działa na szeregu węzłów połączonych w topologii częściowej siatki. Rozróżnia się 2 rodzaje węzłów: centralny i lokalny. W sieci, zwanej konfederacją, występuje jeden węzeł centralny i wiele węzłów lokalnych. 

\subsubsection{Węzeł centralny}
Węzeł centralny może składać się z maksymalnie 4 serwerów fizycznych lub maksymalnie 6 serwerów zwirtualizowanych działających na 4 serwerach fizycznych i na tych zasobach sprzętowych musi zapewniać taką 
niezawodność, że awaria jednego serwera fizycznego nie powoduje utraty funkcjonalności Systemu. Obsługa
awarii pojedynczego serwera musi być tak zrealizowana, że przerwa w działaniu węzła centralnego nie będzie dłuższa niż 10 sekund.

Węzeł centralny ma łączność z Internetem i jest widoczny pod publicznym adresem IP. Węzeł centralny poprzez Internet komunikuje się ze wszystkimi węzłami lokalnymi.

\subsubsection{Węzeł lokalny}
Węzeł lokalny musi być możliwy do uruchomienia na pojedynczym serwerze fizycznym lub zwirtualizowanym z zainstalowanym systemem operacyjnym Debian GNU/Linux w wersji 6.0 lub nowszej i ma łączność z LSS oraz z Internetem. Na styku z Internetem jest widoczny pod publicznym adresem IP. Każdy węzeł lokalny poprzez Internet komunikuje się z węzłem centralnym i wybranymi węzłami lokalnymi.  

%% --------------- Akapit testowy --------------
Testowe odwołanie do sekcji \ref{sec:arch_fiz}. Odwoływać się można do różnych elementów. Np. do rysunków, tabel itp. Jest też możliwość odwołania się do numeru strony, na której obiekt się znajduje. Uwaga dla początkujących: czasem trzeba dokument wyrenderować 2 razy aby w pełni zaktualizowały się odnośniki i wszelkiego rodzaju spisy (np. spis treści).
\footnote{Jeśli renderujemy do PDF'a, to referencje mogą być linkami. Kliknięcie w link będzie powodowało przewinięcie dokumentu do linkowanego obiektu}
%%-----------------------------------------------

\subsection{Bezpieczeństwo sieciowe}
Węzeł centralny oraz węzły lokalne muszą być dwojako zabezpieczone:
\begin{enumerate}
\item wszystkie interfejsy zewnętrzne (interfejsy użytkownika, remote API itp) muszą posiadać zabezpieczenia
pozwalające na kontrolę dostępu do nich za pomocą odpowiednich mechanizmów uwierzytelniania i autoryzowania;
komunikacja z interfejsami zewnętrznymi odbywa się poprzez połączenia szyfrowane; mechanizmy te muszę zapewniać pełne bezpieczeństwo niezależne od zabezpieczenia drugiego,
\item dostęp do interfejsów zewnętrznych jest kontrolowany za pomocą zapory ogniowej serwera udostępniającego
dany interfejs.
\end{enumerate}

\subsection{Konfigurowalność}

TODO Co musi być konfigurowalne (wspólne dla obu podsystemów)

\subsection{Ochrona danych osobowych}

Węzeł centralny nie gromadzi danych osobowych ani danych wrażliwych. Węzły lokalne,
będące pod zarządem jednostek medycznych, u których są zainstalowane mogą
przetwarzać dane osobowe.

\subsection{Rejestracja przebiegu procesów oraz raportowanie}

Przebieg wszystkich procesów realizowanych przez węzły lokalne oraz węzeł 
centralny musi być rejestrowany w sposób umożliwiający: 
\begin{enumerate}
  \item prowadzenie rozliczeń finansowych pomiędzy zlecającymi i konsultantami,
  \item przeprowadzenie audytu poprawności przebiegu procesów,
  \item stwierdzenie kto, kiedy i jakie dane modyfikował, a w przypadku dostępu do danych jednostki zewnętrznej, również fakt przeglądania danych medycznych.
\end{enumerate}

%--------------------------------------------------------------------------------------------
%--------------------------------------------------------------------------------------------
\section[Podsystem Elektronicznego Zlecania Opisów Badań Medycznych]{Wymagana funkcjonalność Podsystemu Elektronicznego Zlecania Opisów Badań Medycznych}

Zadaniem podsystemu jest umożliwienie użytkownikom węzłów lokalnych zlecanie między sobą opisów badań medycznych oraz odbiór gotowych opisów. Zlecenia wysyłane są do węzła centralnego, który dystrybuuje je do węzłów lokalnych konsultantów. Zlecenie stanowi pakiet danych zawierający treść zlecenia oraz pliki będące wynikiem badania, które ma zostać opisane. 

\subsection{Wymagania ogólne}
\begin{enumerate}
\item Co najmniej 3 poziomy dostępu:
        \begin{enumerate}
        \item technik -- może tworzyć zlecenia, ale nie może ich wysłać,
        \item lekarz (zleceniodawca) -- uprawnienia technika + możliwość wysyłania zleceń,
        \item administrator -- ma możliwość zmiany konfiguracji systemu i ingerencji w dane poza produkcyjnym schematem działania.
        \end{enumerate}
\item Routingiem zleceń zajmuje się węzeł centralny.
\item Na węźle centralnym nie pojawiają się dane wrażliwe i osobowe.
\item W celu wymiany danych wrażliwych węzły lokalne komunikują się bezpośrednio między sobą.
\item Komunikaty sygnalizacyjne przechodzą przez węzeł centralny.
\item Węzeł centralny posiada pełną wiedzę zarówno o aktualnie przetwarzanych zleceniach jak i o zleceniach przetworzonych w przeszłości.
\end{enumerate}

\subsection{Definicja zlecenia}

Zlecenie jest dokumentem elektronicznym zawierającym opis zlecenia wraz z niezbędnymi atrybutami oraz listę załączonych dokumentów (również elektronicznych). Załączone dokumenty muszą być identyfikowane w sposób jednoznaczny. Musi istnieć również mechanizm pozwalający na zlokalizowanie dokumentu na podstawie jego identyfikatora lub identyfikatora w połączeniu ze zleceniem. Zamawiający zaleca stosowanie URI do identyfikacji załączonych dokumentów.

Zlecenie powinno posiadać co najmniej następujące atrybuty:
\begin{enumerate*}
\item globalnie unikatowy numer zlecenia
\item zanonimizowane dane o pacjencie, takie jak wiek, czy płeć
\item treść definiująca co ma być opisane
\item określenie odbiorcy zlecenia
\item określenie usługi do której zlecenie jest kierowane
\item określenie priorytetu obsługi
\end{enumerate*}

\subsection{Routing zleceń}

TODO jak ma działać routing zleceń

\subsection{Integracja węzłów lokalnych z lokalnymi systemami szpitalnymi}

Węzeł lokalny musi mieć możliwość komunikacji z lokalnym systemem PACS w celu:
\begin{enumerate}
 \item po stronie zleceniodawcy: pozyskania wyników badania, które ma być opisane,
 \item po stronie konsultanta: przesłania wyników badania na stację diagnostyczną.
\end{enumerate}

Węzeł lokalny musi mieć interfejs HL7 służący do komunikacji z systemami RIS. Interfejs musi mieć funkcjonalność potrzebną do zlecania opisu badań z poziomu systemu RIS oraz przekazywania gotowych opisów do systemu RIS. Interfejs musi być dokładnie opisany, tak aby podmioty trzecie mogły dostosować swoje systemy do korzystania z tego interfejsu. Dodatkowo interfejs w miarę możliwości powinien być tak zaprojektowany, aby nie sprawiał zbędnych trudności przy integracji z takimi systemami jak: TODO lista systemów, 

Pozyskiwanie danych z systemu PACS musi odbywać się na 2 sposoby:
\begin{enumerate}
  \item aktywne odpytanie systemu PACS i pobranie plików do bufora na węźle lokalnym
  \item pasywne oczekiwanie na dostarczenie plików przez PACS lub inny węzeł DICOM; dostarczone pliki zapisywane są w buforze na węźle lokalnym
\end{enumerate}

\subsection{Interfejs użytkownika}

Wymagany jest webowy interfejs użytkownika do realizacji wszystkich podstawowych funkcji systemu. Możliwe jest zastosowanie również dodatkowych niewebowych interfejsów użytkownika. Każdy z węzłów lokalnych oraz węzeł centralny muszą mieć swoje własne interfejsy użytkownika.

Wymagania dla webowego interfejsu użytkownika:

\begin{enumerate}
    \item neutralność technologiczna zapewniona poprzez:
	  \begin{enumerate}
	  \item poprawne działanie na najnowszej na dzień ogłoszenia przetargu wersji wieloplatformowych przeglądarek WWW Mozilla Firefox oraz Google Chrome
	  \item realizację wszystkich podstawowych funkcji interfejsu użytkownika bez wymagania jakichkolwiek dodatków do przeglądarek, czy używania funkcji przeglądarek specyficznych dla określonego systemu operacyjnego na którym przeglądarka działa.
	  \end{enumerate}
    \item sprawność działania zapewniona poprzez:
	  \begin{enumerate}
	  \item takie zaprojektowanie interfejsu, aby zminimalizować ilość interakcji z użytkownikiem -- minimalizacja ilości kliknięć wymaganych do wykonania określonej czynności oraz ilość wprowadzanych ręcznie przez użytkownika danych
	  \item takie zaprojektowanie interfejsu aby zminimalizować czas reakcji interfejsu na akcję użytkownika - interfejs użytkownika musi sprawnie działać [określić, co to znaczy sprawnie działać] w przeglądarce uruchomionej na komputerze z procesorem o wydajności KKK i 360 MB pamięci RAM dostępnej dla przeglądarki
	  \end{enumerate}
\end{enumerate}

Podstawowe funkcje interfejsu użytkownika:
\begin{enumerate}
  \item Tworzenie zlecenia:
      \begin{enumerate}
      \item przy tworzeniu zlecenia użytkownik musi mieć możliwość dostarczenia kompletnej definicji zlecenia (patrz opis treści zlecenia)
      \item użytkownik musi mieć możliwość załączenia do zlecenia plików z wynikami badań oraz innych plików (np. skan  skierowania na badanie); wymagana jest obsługa 3 sposobów dostarczania plików: wybór pliku z zasobów dyskowych dostępnych w systemie operacyjnym, na którym działa przeglądarka, wybór pliku znajdującego się w buforze plików DICOM na węźle lokalnym, uruchomienie wyszukiwarki plików w systemie PACS, a następnie pobranie wybranych przez użytkownika plików do bufora i automatyczne załączenie ich do zlecenia
      \end{enumerate}
  \item Wysyłanie zlecenia:
\end{enumerate}

\subsection{Schemat pracy}
\begin{enumerate}
\item Użytkownik łączy się przeglądarką WWW z lokalnym interfejsem Systemu, loguje się do niego i ma możliwość:
        \begin{enumerate}
        \item przejrzenia stanu aktualnych (będących w trakcie przetwarzania) zleceń,
        \item modyfikacji aktualnych, niezakończonych zleceń,
        \item przeglądania zleceń archiwalnych,
        \item stworzenia nowego zlecenia,
        \item wysłania gotowego zlecenia,
        \item realizacji zlecenia (opisania badania).
        \end{enumerate}
\item W ramach tworzenia/modyfikacji zlecenia użytkownik może:
        \begin{enumerate}
        \item wybrać pliki DICOM (pogrupowane np. wg. StudyUID, albo nazwy pacjenta) znajdujące się w buforze na węźle (pliki do bufora trafiają w wyniku wysłania ich do węzła z poziomu systemu PACS)
        \item wyszukać i pobrać pliki DICOM bezpośrednio z PACS
        \item wysłać poprzez interfejs web plik istniejący na terminalu użytkownika
        \item zmodyfikować wszystkie istotne na tym etapie parametry zlecenia
        \end{enumerate}
\end{enumerate}

\subsection{Konfigurowalność}

TODO Co musi być konfigurowalne w systemie zlecania 
TODO konfigurowalność interfejsu HL7

%--------------------------------------------------------------------------------------------
%--------------------------------------------------------------------------------------------
\section[Podsystem Elektronicznej Wymiany Danych Medycznych]{Wymagana funkcjonalność Podsystemu Elektronicznej Wymiany Danych Medycznych}

\subsection{Definicje}
\begin{description}
\item[użytkownik] Uprawniony pracownik jednostki medycznej posiadającej węzeł lokalny, który chce pozyskać dane
      medyczne dostępne w innych jednostkach medycznych.
\item[dane medyczne] Wyniki badań medycznych pacjentów oraz ich opisy przechowywane w LSS (zwykle będą to pliki DICOM).
\item[identyfikator pacjenta] Identyfikator pacjenta używany w LSS. Identyfikator pacjenta zawsze występuje w kontekście
      konkretnego LSS, a bez niego nie ma znaczenia.
\item[baza informacji o danych medycznych] Baza danych zawierająca informacje o danych medycznych dostępnych poprzez węzły lokalne.
      Baza zawiera identyfikatory pacjentów oraz odnośniki do powiązanych z tymi identyfikatorami danych medycznych.
\item[baza powiązań identyfikatorów] Baza danych wiążąca identyfikatory pacjentów z różnych LSS.
\end{description}

\subsection{Schemat działania}

\begin{enumerate}
  \item Użytkownik uwierzytelnia się w systemie i uzyskuje dostęp do interfejsu użytkownika.
  \item Użytkownik wprowadza do węzła lokalnego dane osobowe pacjenta, którego dane medyczne są poszukiwane.
  \item Węzeł lokalny wysyła zapytanie do węzła centralnego i oczekuje na wyniki.
  \item Po otrzymaniu wyników, węzeł lokalny prezentuje użytkownikowi listę dostępnych danych medycznych.
  \item Użytkownik wybiera, które dane medyczne chce przejrzeń, a następnie pozyskuje je bezpośrednio od
        zdalnego węzła lokalnego, który je udostępnia.
\end{enumerate}

\subsection{Opis węzła centralnego}
Zadania węzła centralnego:
\begin{enumerate}
\item utrzymanie bazy informacji o danych medycznych,
\item utrzymanie bazy powiązań identyfikatorów,
\item routing zapytań,
\item przyjmowanie od węzłów lokalnych informacji o dostępnych danych medycznych,
\item przyjmowanie od węzłów lokalnych informacji o powiązaniach identyfikatorów pacjentów,
\end{enumerate}

\subsection{Opis węzła lokalnego}
Zadania węzła lokalnego:
\begin{enumerate}
\item Udostępnianie interfejsu użytkownika niezbędnego do prowadzania danych osobowych pacjentów
      oraz przeglądania pozyskanych danych medycznych.
\item Informowanie węzła centralnego o dostępnych w LSS danych medycznych (budowa bazy informacji o danych medycznych)
\item Informowanie węzła centralnego o odkrytych powiązaniach między identyfikatorami pacjentów
\item Udostępnianie danych medycznych uprawnionym zdalnym węzłom lokalnym

% tutaj skończyłem - seweryn

\item komunikacja z istniejącymi lokalnymi systemami szpitalnymi:
  \begin{enumerate}
  \item należy założyć wykorzystanie w szpitalach oprogramowania pochodzącego od różnych dostawców i mogą
        się pojawić różnice nawet zapisie danych osobowych w plikach DICOM. W tym celu trzeba przewidzieć
        w tworzonym oprogramowaniu interfejs do tworzenia standaryzacji danych osobowych postaci jednolitej
        dla systemu. Same dane nie będą jednak przechowywane w węźle lokalnym – posłużą do stworzenia odwzorowania
        na jednoznaczny identyfikator pacjenta na poziomie systemu
  \item ma za zadanie przyjmować od oprogramowania szpitalnego informacje o zajściu zdarzenia dot.
        pacjenta (np. poprzez HL7)
  \item ma za zadanie pośredniczyć w udostępnianiu danych medycznych do innych węzłów lokalnych realizując
        ich zapytanie. W tym celu po zidentyfikowaniu pacjenta wysyła do takiego systemu linki do lokalnych
        plików (np. DICOM) z jego danymi medycznymi
  \end{enumerate}

\item komunikacja z węzłem centralnym
  \begin{enumerate}
  \item przekazywane są powiązania informacje o jednoznacznych identyfikatorach pacjentów
  \item opcjonalnie następuje synchronizacja z węzłem centralnym, o której mowa w pkt. 4.2.4.
  \end{enumerate}

\item komunikacja z innymi węzłami lokalnymi
  \begin{enumerate}
  \item udostępnianie danych medycznych pacjentów innym węzłom lokalnym jak opisane w pkt.  4.3.1.3.
  \item pobieranie danych z innych węzłów lokalnych
  \end{enumerate}
\end{enumerate}

Ponadto komunikacja pomiędzy węzłami lokalnymi między sobą, jak i pomiędzy nimi, a węzłem głównym musi być zabezpieczona przez zestawienie bezpiecznego tunelu VPN na poziomie sieci komputerowej (site-to-site, 'przezroczysty' VPN) lub jeśli to nie jest możliwe na poziomie węzła (client-to-site VPN)

Transmisja danych musi zapewniać integralność, poufność i autentyczność przesyłanych danych. Można to zapewnić przez bezpośredni tunel VPN pomiędzy węzłami lokalnymi.

\end{document}
