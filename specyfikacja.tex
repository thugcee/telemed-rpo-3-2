\documentclass[a4paper]{article}

\usepackage{polski}
\usepackage[utf8x]{inputenc}
\usepackage[OT4]{fontenc}
\usepackage[polish]{babel}
\usepackage[pdftex]{graphicx}
\usepackage[pdftex]{hyperref}

\title{Specyfikacja wymagań dla systemu telemedycznego realizowanego w ramach projektu XXX}
\author{Zespół Projektowy}
\date{03/03/2012}

\begin{document}

Opis funkcjonalny Systemu

\section{Wstęp}

Od strony funkcjonalnej system składa się z 2 niezależnych podsystemów działających na 
tej samej platformie sprzętowej: Podsystem Elektronicznego Zlecania Opisów Badań 
Medycznych oraz Podsystem Elektronicznej Wymiany Danych Medycznych. 

\section{Założenia wspólne dla obu podsystemów}

\subsection{Architektura fizyczna }

System działa na szeregu węzłów połączonych w topologii częściowej siatki. Rozróżnia się 
2 rodzaje węzłów: centralny i lokalny. W sieci (konfederacji?) występuje jeden węzeł 
centralny i wiele węzłów lokalnych. Każdy węzeł lokalny komunikuje się z węzłem 
centralnym i wybranymi przez siebie węzłami lokalnymi. Węzeł lokalny musi być możliwy 
do uruchomienia na pojedynczym serwerze fizycznym lub zwirtualizowanym z zainstalowanym 
systemem operacyjnym Debian GNU/Linux w wersji 6.0 lub nowszej. Węzeł centralny może 
składać się z maksymalnie 4 serwerów fizycznych lub maksymalnie 6 serwerów zwirtualizowanych. 

\subsection{Ochrona danych osobowych}

System nie gromadzi danych osobowych ani danych wrażliwych.

\subsection{Rejestracja przebiegu procesów oraz raportowanie}

Przebieg wszystkich procesów realizowanych przez węzły lokalne oraz węzeł 
centralny musi być rejestrowany w sposób umożliwiający: 
\begin{itemize}
  \item prowadzenie rozliczeń finansowych pomiędzy zlecającymi i konsultantami,
  \item przeprowadzenie audytu poprawności przebiegu procesów.
\end{itemize}

\section{Założenia co do funkcjonalności Podsystemu Elektronicznego Zlecania Opisów Badań Medycznych}

Zadaniem podsystemu jest umożliwienie użytkownikom węzłów lokalnych zlecanie między sobą opisów 
badań medycznych oraz odbiór gotowych opisów. Zlecenia wysyłane są do węzła centralnego, który 
dystrybuuje je do węzłów lokalnych konsultantów. Zlecenie stanowi pakiet danych zawierający treść 
zlecenia oraz pliki będące wynikiem badania, które ma zostać opisane. 

\subsection{Definicje}

\begin{description}
\item[Zleceniodawca] użytkownik węzła lokalnego zlecający do systemu wykonanie opisu badania
\item[Konsultant] użytkownik węzła lokalnego realizujący zlecenie opisu badania, które pojawiło się na jego węźle
\end{description}

\subsection{Integracja węzłów lokalnych z lokalnymi systemami szpitalnymi}

Węzeł lokalny musi mieć możliwość komunikacji z lokalnym systemem PACS w celu:
\begin{itemize}
 \item po stronie zleceniodawcy: pozyskania wyników badania, które ma być opisane,
 \item po stronie konsultanta: przesłania wyników badania na stację diagnostyczną.
\end{itemize}

Węzeł lokalny musi mieć interfejs HL7 służący do komunikacji z systemami RIS. Interfejs musi mieć funkcjonalność 
potrzebną do zlecania opisu badań z poziomu systemu RIS oraz przekazywania gotowych opisów do systemu RIS. 
Interfejs musi być dokładnie opisany, tak aby podmioty trzecie mogły dostosować swoje systemy do korzystania 
z tego interfejsu. Dodatkowo interfejs w miarę możliwości powinien być tak zaprojektowany, aby nie sprawiał 
zbędnych trudności przy integracji z takimi systemami jak: ........

Pozyskiwanie danych z systemu PACS musi odbywać się na 2 sposoby:
\begin{itemize}
  \item aktywne odpytanie systemu PACS i pobranie plików do bufora na węźle lokalnym
  \item pasywne oczekiwanie na dostarczenie plików przez PACS lib inny węzeł DICOM; dostarczone 
	pliki zapisywane są w buforze na węźle lokalnym
\end{itemize}

\subsection{Interfejs użytkownika}

Wymagany jest webowy interfejs użytkownika do realizacji wszystkich podstawowych funkcji systemu. 
Możliwe jest zastosowanie również dodatkowych niewebowych interfejsów użytkownika. Każdy z węzłów 
lokalnych oraz węzeł centralny muszą mieć swoje własne interfejsy użytkownika.

Wymagania dla webowego interfejsu użytkownika:

\begin{itemize}
    \item neutralność technologiczna zapewniona poprzez:
	  \begin{itemize}
	  \item poprawne działać na najnowszej na dzień ogłoszenia przetargu wersji
		  wieloplatformowych przeglądarek WWW Mozilla Firefox oraz Google Chrome
	  \item realizację wszystkich podstawowych funkcji interfejsu użytkownika
		  bez wymagania jakichkolwiek dodatków do przeglądarek, czy używania funkcji przeglądarek 
		  specyficznych dla określonego systemu operacyjnego na którym przeglądarka działa.
	  \end{itemize}
    \item sprawność działania zapewniona poprzez:
	  \begin{itemize}
	  \item takie zaprojektowanie interfejsu, aby zminimalizować ilość interakcji z 
		użytkownikiem -- minimalizacja ilości kliknięć wymaganych do wykonania określonej 
		czynności oraz ilość wprowadzanych ręcznie przez użytkownika danych
	  \item takie zaprojektowanie interfejsu aby zminimalizować czas reakcji interfejsu na 
		akcję użytkownika - interfejs użytkownika musi sprawnie działać [określić, co to 
		znaczy sprawnie działać] w przeglądarce uruchomionej na komputerze z procesorem o 
		wydajności KKK i 360 MB pamięci RAM dostępnej dla przeglądarki 
	  \end{itemize}
\end{itemize}

Podstawowe funkcje interfejsu użytkownika:
\begin{itemize}
  \item Tworzenie zlecenia:
      \begin{itemize}
      \item przy tworzeniu zlecenia użytkownik musi mieć możliwość dostarczenia kompletnej definicji zlecenia (patrz opis treści zlecenia)
      \item użytkownik musi mieć możliwość załączenia do zlecenia plików z wynikami badań oraz innych plików (np. skan 
	    skierowania na badanie); wymagana jest obsługa 3 sposobów dostarczania plików: wybór pliku z zasobów 
	    dyskowych dostępnych w systemie operacyjnym, na którym działa przeglądarka, wybór pliku znajdującego się 
	    w buforze plików DICOM na węźle lokalnym, uruchomienie wyszukiwarki plików w systemie PACS, a następnie 
	    pobranie wybranych przez użytkownika plików do bufora i automatyczne załączenie ich do zlecenia
      \end{itemize}
  \item Wysyłanie zlecenia:
\end{itemize}

\section{Wymagana funkcjonalność Podsystemu Elektronicznej Wymiany Danych Medycznych}

\subsection{Definicje}
\begin{description}
\item[poszukujący informacji o pacjencie] – użytkownik węzła lokalnego, który chce pozyskać ewentualne dane o pacjencie
	składowane w innych Szpitalnych Systemach Informatycznych 
\end{description}

\subsection{Opis węzła Centralnego}
Węzeł ma za zadanie:
\begin{itemize}
\item utrzymywać bazę danych zawierającą jednoznaczne identyfikatory pacjentów połączone z informacją o węźle lokalnym, z którego one pochodzą
\item nie przechowuje danych osobowych, ani danych wrażliwych. Ww. identyfikatory bedą odwzorowaniami identyfikatorów pacjentów sosowanych w lokalnych szpitalnych systemach informatycznych. Odwzorowania te będą wykonywane przez oprogramowanie węzłów lokalnych
\item obsługiwać zapytania kierowane z wezłów lokalnych poszukujących danych medycznych o pacjencie. Obsługa takigo zapytania ma polegać na przesłaniu w odpowiedzi informacji o wszystkich lokalnych węzłach, które posiadają informacje o zdarzeniach dotyczących tego pacjenta. Dalsza wymiana informacji odbywa się bez udziału węzła centralnego – bezpośrednio pomiędzy węzłami lokalnymi
\item jako opcja węzeł centralny synchronizuje swoją bazę danych odwzorowań (pacjentów) z węzłami loalnymi, żeby przyspieszyć proces realizacji zapytań
\end{itemize}

\subsection{Opis węzła lokalnego}
Węzeł ma spełniać funkcje:
\begin{itemize}
\item komunikacja z isniejącymi lokalnymi systemami szpitalnymi:
  \begin{itemize}
  \item należy założyć wykorzystanie w szpitalach oprogramowania pochodzącego od różnych dostawców i mogą 
	  się pojawić różnice nawet zapisie danych osobowych w plikach DICOM. W tym celyu trzeba przewidzieć 
	  w tworzonym oprogramowaniu interfejs do tworzenia stadaryzacji danych osobowych postaci jednolitej 
	  dla systemu. Same dane nie będą jednak przechowywane w węźle lokalnym – posłużą do stworzenia odwzorowania 
	  na jednoznaczny identyfikator pacjenta na poziomie systemu
  \item ma za zadanie przyjmować od oprogramowania szpitalnego informacje o zajściu zdarzenia dot. pacjenta (np. poprzez HL7)
  \item ma za zadanie pośredniczyć w udostępnianiu danych medycznych do innych węzłów lokalnych realizując 
	  ich zapytanie. W tym celu po zidetyfikowniu pacjenta wysyła do takiego systemu linki do lokalnych 
	  plików (np. DICOM) z jego danymi medycznymi
  \end{itemize}

\item komunikacja z węzłem centralnym
  \begin{itemize}
  \item przekazywane są powiązania informacje o jednoznacznych identyfiaktorach pacjentów
  \item opcjonalnie następuje sychnchronizacja z węzłem centralnym, o kótrej mowa w pkt. 4.2.4.
  \end{itemize}

\item komunikacja z innymi węzłami lokalnymi
  \begin{itemize}
  \item udostępnianie danych medycznych pacjentów innym węzłom lokalnym jak opisane w pkt.  4.3.1.3.
  \item pobieranie danych z innych węzłów lokalnych
  \end{itemize}
\end{itemize}

Ponadto komunikacja pomiędzy węzłami lokalnymi między sobą, jak i pomiędzy nimi, a węzłem głównym musi 
być zabezpieczona przez zestawienie bezpiecznego tunelu VPN na poziomie sieci komputerowej (site-to-site, 
'przeroczysty' VPN) lub jeśli to nie jest możliwe na poziomie węzła (client-to-site VPN)

Transmisja danych musi zapewniać intgeralność, poufność i autentyczność przesyłanych danych. 
Można to zpewnić przez bezpośredni tunel VPN pomiędzy węzłami lokalnymi.

\end{document}
